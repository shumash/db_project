% THIS IS AN EXAMPLE DOCUMENT FOR VLDB 2012
% based on ACM SIGPROC-SP.TEX VERSION 2.7
% Modified by  Gerald Weber <gerald@cs.auckland.ac.nz>
% Removed the requirement to include *bbl file in here. (AhmetSacan, Sep2012)
% Fixed the equation on page 3 to prevent line overflow. (AhmetSacan, Sep2012)

\documentclass{vldb}
\usepackage{graphicx}
\usepackage{balance}
\usepackage{ dsfont }
% for  \balance command ON LAST PAGE  (only there!)


\begin{document}

% ****************** TITLE ****************************************

\title{Rasterized Image Databases with LSH\\
for Compression, Search and Duplicate Detection}

% possible, but not really needed or used for PVLDB:
%\subtitle{[Extended Abstract]
%\titlenote{A full version of this paper is available as\textit{Author's Guide to Preparing ACM SIG Proceedings Using \LaTeX$2_\epsilon$\ and BibTeX} at \texttt{www.acm.org/eaddress.htm}}}

% ****************** AUTHORS **************************************

% You need the command \numberofauthors to handle the 'placement
% and alignment' of the authors beneath the title.
%
% For aesthetic reasons, we recommend 'three authors at a time'
% i.e. three 'name/affiliation blocks' be placed beneath the title.
%
% NOTE: You are NOT restricted in how many 'rows' of
% "name/affiliations" may appear. We just ask that you restrict
% the number of 'columns' to three.
%
% Because of the available 'opening page real-estate'
% we ask you to refrain from putting more than six authors
% (two rows with three columns) beneath the article title.
% More than six makes the first-page appear very cluttered indeed.
%
% Use the \alignauthor commands to handle the names
% and affiliations for an 'aesthetic maximum' of six authors.
% Add names, affiliations, addresses for
% the seventh etc. author(s) as the argument for the
% \additionalauthors command.
% These 'additional authors' will be output/set for you
% without further effort on your part as the last section in
% the body of your article BEFORE References or any Appendices.

\numberofauthors{4} %  in this sample file, there are a *total*
% of EIGHT authors. SIX appear on the 'first-page' (for formatting
% reasons) and the remaining two appear in the \additionalauthors section.

\author{
% You can go ahead and credit any number of authors here,
% e.g. one 'row of three' or two rows (consisting of one row of three
% and a second row of one, two or three).
%
% The command \alignauthor (no curly braces needed) should
% precede each author name, affiliation/snail-mail address and
% e-mail address. Additionally, tag each line of
% affiliation/address with \affaddr, and tag the
% e-mail address with \email.
%
% 1st. author
\alignauthor
Zoya Bylinskii\\
       \affaddr{MIT CSAIL}
% 2nd. author
\alignauthor
Maria Shugrina\\
\affaddr{MIT CSAIL}
\and  % use '\and' if you need 'another row' of author names
% 3rd. author
\alignauthor
Andrew Spielberg\\
       \affaddr{MIT CSAIL}
% 4th. author
\alignauthor
Wei Zhao\\
\affaddr{MIT CSAIL}
}
\date{\today}
% Just remember to make sure that the TOTAL number of authors
% is the number that will appear on the first page PLUS the
% number that will appear in the \additionalauthors section.


\maketitle

\begin{abstract}
TODO.
\end{abstract}




%\section{Introduction}

\section{Problem Statement}


We formally define the problem as follows.  Consider an input set of $i$ images, and without loss of generality, assume each image is square and is composed of $m^2$ pixels.  We wish to store our set of images in a database using lossy compression such that we minimize the total space used by the database subject to certain \emph{quality} constraints based on similarity metrics.  There are many possible formulations of these quality constraints; we detail those that we considered for this paper in section TODO.  Beyond mathematical metrics, subjective methods are also interesting to consider for formulating the quality constraints; one could imagine a scenario in which image quality is assessed by humans through a crowdsourced system, perhaps using an engine such as Amazon's Mechanical Turk TODO: cite.  We consider such subjective similarity metrics outside the scope of this paper and focus on the mathematical metrics for now.

We wish to construct a database that trades off minimizing the amount of required space with maximizing read and write speeds of the data.  In the following sections, we discuss models for estimating these values.  We construct our compressed representation in the following way: First, we segment each image into a certain number of image patches.  Next, we store a set of precomputed image patch "exemplars" in an auxiliary table.  Finally, rather than represent each patch in each image explicitly, we approximate it by instead storing a pointer to the "most similar" exemplar.  We discuss similarity later in section TODO as we discuss our quality constraints.  For this paper, we consider all patches to be the same size and each image to contain the same number of patches.

\section{Related Work}\label{sec:related}

Image databases, in particular rasterized.

Image compression, in particular JPEGs.

Image similarity functions.

Image quality functions.

Hashing, in particular~\cite{LSH:Andoni}.

\section{Analysis}
Assume for now that we choose to store $p$ patches in our auxiliary table.  In practice, we choose $p$ to be a function $p \colon Function \to \mathds{N}$ which maps from our similarity metric to a number of patches to store.  Assume also that each patch is square and composed of $n^2$ pixels, where $n$ is a user defined parameter.  We further assume that each pixel requires 8 bytes to store  and that each pointer is 8 bytes (a standard integer for a 64-bit system).  Under this "image only" scheme, in the case where we have $i$ images, the cost $c_i$ to store all the images in our database is:

\begin{equation}
	c_i(i, m) = 8  i  m^2
\end{equation}

In the case where we store pointers to patches, we have two tables: one table to store pointers to image patch exemplars, and a second table to store the exemplar data themselves.  Under this "patch pointer"scheme, in the case where we have $i$ images and $p$ patches, the cost $c_p$ to store all the images in our database is:

\begin{equation}
	c_p(i, p, m, n) = 8 i (\frac{m}{n})^2 + 8  p  n^2
\end{equation}.

The first term is the cost of storing the pointer data, while the second term is the cost of storing the patch exemplars themselves.




% NOTE: include this input to see all the latex tips
%\input{latex_tips}

\bibliographystyle{abbrv}
\bibliography{db_report}

\end{document}
