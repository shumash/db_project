\section{Conclusion}

\begin{edit}
Summarize paper contributions
\end{edit}

\subsection{Future Extensions}
\label{sec:futureext}

In this paper we considered a simple implementation of a patch-based image database compression scheme, where the images and patches were square and of fixed sizes. Patches were sampled in a regular, non-overlapping grid from each image.  Alternative approaches include more flexible, context-aware, patch-sampling techniques. For instance, the patch granularity for sampling large homogenous sky and field regions may be different from the one used for sampling highly-textured regions like objects and structures (trees, buildings, people, etc.). Similarly, patches that do not cross object boundaries are likely to lead to less artifacts in future reconstructions. For this, approaches like Selective Search \cite{UijlingsIJCV2013} that localize image regions likely to contain objects, may prove promising for sampling patches.

\begin{edit}

task-aware threshold selection

patch hierarchies
\end{edit}