\section{Evaluation and Discussion}\label{sec:performance}

Once the quality threshold is chosen, the key differentiator
between the speed of database construction and the quality of
reconstructed images is the hashing strategy.
We constructed 3 databases on the same set of
10,000 images sampled
from all categories of the SUN database, with image size
set to 500, and patch size set to 25, using the following
hashing strategies for Near Neighbor(NN) search:\\
\begin{enumerate}
\item \textbf{Naive NN}: 10 random projection vectors sampled from unit Normal with
uniform bin size (outliers truncated)
\item \textbf{PCA NN}: 10 first principla components as projection vectors with
bin size adapted to the distribution of projections
\item \textbf{PCA + U NN}: nearly uniform patches are hashed using Luv color
quantization into 864 bins, and non-uniform patches are handled with PCA NN
\end{enumerate}
We compare and contrast the effects of hashing on performance in
Sec.~\ref{ssec:nn-eval}, and detail a qualitative evaluation of
the results with a small user study in ~\ref{ssec:qual-eval}.

\subsection{Evaluating NN Strategies}\label{ssec:nn-eval}



\begin{edit}
Zoya: fill in the following\\
Graph1: fig\_NN/dict\_growth.jpg (img vs. num patches)\\
Graph2: fig\_NN/bin\_cover.jpg (bins needed to cover percent of patches)
Graph3: fig\_NN/ave\_patches\_per\_img.jpg\\
1. explain that naive never finished \\
2. explain Graph 1 \\
3. explain why Naive has fewer patches \\
4. explain why Naive ground to a halt (see Graph2: most of the data falls into
few bins - PCA helps with this by projecting data onto directions
of maximal variance to differentiate between patches) \\
5. note that the DB is not large enough to be conclusive
(personal photo collections are larger)\\
6. state that Graph3 shows promise - average \# of patches/image is going down
(also explain why there is a difference between the plots (or MAsha can do that)
\end{edit}
%FR - found ratio: Count(good match found)/Count(good match not found)

% Table
\begin{table*}
\begin{tabular}{ | l | l | l | l | l | l | l | l | }
\hline
& time & \#patches & \#bins & FR & Stime & FPQ & IRQ \\
\hline
Naive NN & & & & & & & \\
PCA NN & 7h27min & 3,309,583 & 2,751,235 & & & &  \\
PCA + U NN & & & & & & &\\
\hline
\end{tabular}
\caption{Results on 10,000 images samples from all
the categories of the SUN database, where the rows
are for naive projection hashing, PCA-based hashing and
PCA-based hashing combine with uniform patch hashing.}
\label{tb:nn-res}
\end{table*}

PCA NN: Finished batch-upload of 10000 files in 26827397
NAIVE NN: 7:35AM - 1:50AM for 4367 images



\subsection{Quantitative Quality}\label{ssec:quant}
\begin{edit}
TODO
\end{edit}


\subsection{Qualitative Qualitative}\label{ssec:qual}
\begin{edit}
TODO
\end{edit}

\begin{edit}
If time allows, insert sample image reconstructions from different categories
\end{edit}
