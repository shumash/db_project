\section{Method}\label{sec:method}

To limit the complexity, we assume
each image to be a square of $m^2$ pixels, and for all patches to
be squares of the same size.
We formally describe our method in Sec.~\ref{ssec:overview}. To summarize,
we first segment each image into patches and store them in the \texttt{patches} database.
Only patches sufficiently different
from all the patches in \texttt{patches} are stored (See Sec.~\ref{ssec:sim}).
Instead of storing each image explicitly, we approximate it by storing pointers to the patches
approximating its original patches. In Sec.~\ref{ssec:lsh}, we describe the
search algorithm used to quickly retrieve similar patches, and in Sec.~\ref{ssec:reconst}
we detail the indices that make image reconstruction faster. Finally, in
Sec.~\ref{ssec:impl} we provide implementation details.

\subsection{Overview}\label{ssec:overview}

Our method is governed by the following parameters:
\begin{itemize}
\item $n$ - width and height of all patches\footnote{We assume that $m \mod n$ is $0$.}
\item $S$ - similarity function from two $n \times n$ images to $\mathbb{R}$
\item $T$ - similarity threshold
\end{itemize}
The $n \times n$ patches are stored as byte data in the \texttt{patches} table.
To allow image reconstruction, we also store $(m / n)^2$ patch pointers for
each image in the \texttt{patch\_pointers} table. The full schema looks as follows:
\begin{verbatim}
patches(id int PRIMARY KEY,
        patch bytea);

images(imgid int PRIMARY KEY);

patch_pointers(imgid int REFERENCES images(imgid),
               patch_id int REFERENCES patches(id),
               x int,
               y int);
\end{verbatim}
where \texttt{patch\_pointers.x} and \texttt{patch\_pointers.y}
refer to the left top corner location of each patch in the image.

Given these tables, inserting an image into the database proceeds as follows:

\begin{algorithm}
    \caption{Insert Image $I$ into database}
    \label{alg:insert}
\begin{algorithmic}[1]
\State $Patches \leftarrow $ \texttt{CutIntoPatches}(I, patch\_size=$n$)
\For{$P$ in $Patches$}
\State $SimPat \leftarrow $\texttt{FindLikelySimilarPatches}($P$)
\State $P_{closest} \leftarrow $ $argmin \{ S(P, P_i) \}$
\If $S(P, P_i) > T$
\State \texttt{insert} $P$ into \texttt{patches}
\EndIf
\EndFor
\vspace{3mm}
\end{algorithmic}
\end{algorithm}
Sections \ref{ssec:sim} and \ref{ssec:lsh} detail similarity measure and
finding patches that are likely to be similar.

In order to reconstruct an image from that patches, we run the following
procedure:



\subsection{Patch Similarity}\label{ssec:sim}
There are many image similarity metrics that have been developed for
images (See~\cite{yasmin2013use} for a good survey.), and
our method is applicable to any metric that involves Euclidean
distance over image features, its stacked color channel pixel values
being the simplest case.

For the purpose of this project, we choose to use squared Euclidean
distance over (CIE)LUV color space.
Given two $n \times n$ patches $P_1$ and $P_2$, we evaluate similarity $S$
per color channel $i$ as follows:

\begin{displaymath}
S(P_1, P_2, i) = \frac{||P_1(i) - P_2(i)||^2}{n^2}
\end{displaymath}
where $||\cdot||$ denotes standard Euclidean norm.
We normalize by the dimensionality of the space to allow us to keep the
similarity threshold independent of the patch size. See section \ref{sec:simthresh} for more details.

The use of Euclidean distance for patch similarity
allows us to use Locality Sensitive Hashing to retrieve patches
that are likely to be similar, as detailed in the
next section.

\subsection{LSH for Patches}\label{ssec:lsh}

\subsection{Image Reconstruction}\label{ssec:reconst}

In order to reconstruct images quickly, we

\subsection{Implementation}\label{ssec:impl}
We used $postgresql$ to construct our database, and used
Java API to talk to the database from a custom executable. Locality
sensitive hashing, image segmentation and reconstruction were
all implemented in Java, and used to construct a hash table
on patches in $postbresql$.

Our code is available at:
\begin{verbatim}
https://github.com/shumash/db_project
\end{verbatim}
