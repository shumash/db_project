\section{Method}\label{sec:method}

To limit the complexity, we assume
each image to be a square of $m^2$ pixels, and for all patches to
be squares of the same size.
We formally describe our method in Sec.~\ref{ssec:overview}. To summarize,
we first segment each image into patches and store them in the \texttt{patches} database.
Only patches sufficiently different
from all the patches in \texttt{patches} are stored (See Sec.~\ref{ssec:sim}).
Instead of storing each image explicitly, we approximate it by storing pointers to the patches
approximating its original patches. In Section~\ref{ssec:lsh}, we describe the
search algorithm used to quickly retrieve similar patches, and in Section~\ref{ssec:reconst}
we detail the indices that make image reconstruction faster.

\subsection{Overview}\label{ssec:overview}

\subsection{Patch Similarity}\label{ssec:sim}

\subsection{LSH for Patches}\label{ssec:lsh}

\subsection{Image Reconstruction}\label{ssec:reconst}
