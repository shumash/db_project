\section{Related Work}\label{sec:related}

Lossless image compression techniques like \emph{area coding} or \emph{Huffman coding} give quality guarantees but do not always give sufficient savings (especially for unstructured images). Lossy image compression algorithms, on the other hand, are quite popular and used for most applications dealing with natural images, including storage of image datasets, and web transmission. Often, lossy compression depends on a quantization of image regions (e.g. all pixels in a block of the image receive the same color value). For instance, the \emph{JPEG compression} standard partitions an image into non overlapping blocks and applies a discrete cosine transform to each block, quantizing the resulting coefficients. \emph{Fractal compression} techniques \cite{Jacquin} apply an iterative algorithm to encode images as fractal codes, capable of representing different parts of the images at different levels of detail. This algorithm makes use of self-similarities and is computationally expensive. However, due to fast decoding, it can be used for file downloads. All of these approaches, however, operate on a \emph{within-image} basis, compressing an image using either a pre-specified dictionary (quantization) or using within-image similarity. Thus, these approaches provide constant savings for image databases of arbitrary sizes.

In the age of big data, when the amount of images in image collections grows at enormous rates, a compression scheme that scales with the database size seems most appropriate. Thus, we are interested in compression schemes that take into account \emph{across-image} redundancy, not just \emph{within-image} redundancy. 

Some of our inspiration comes from \emph{raster databases}, which encode images (often of geospatial data) as a set of smaller images/regions with locations in the original large image. Such a set-up is convenient for transmission or data loading, as it is possible to load and process only parts of the image at a time. Geodatabases, such as \emph{ArcGIS} are set-up this way \cite{ArcGIS}.

We combine the idea of raster databases with the idea of image compression. Patch-based image representations are also used in computer vision for various tasks, including image matching \cite{Brown05}, object recognition \cite{vashist2006dps}, and image processing \cite{Barnes2009}. Considering images as collections of patches allows for matching without rigid spatial constraints. In other words, such approaches can often find approximate image matches, such as different viewpoints of the same scene. Patch-based approaches have thus proven to be sufficient for scene recognition applications. We adapt a patch-based approach as well. Even very coarse-grained patches can provide sufficient visual information for scene recognition purposes. Thus, if we reconstruct images from approximately-similar coarse-grained patches, they will still provide sufficient input to scene classification algorithms, for instance.